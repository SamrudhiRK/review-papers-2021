A successful attempt to detect gravitational waves was made by LIGO. It stands for Laser Interferometer Gravitational wave Observatory), and was founded by Reiner Weiss (MIT) , Barry Barish (Caltech) and Kip Thorne (Caltech). A joint - initiative to establish LIGO was taken by MIT and Caltech, and was funded by the National Science Foundation(NSF). The construction of the two detection sites at remote locations of Hanford and Livingston, commenced in 1994 and ended in 1999. \\

Two sites were built instead of one, in order to avoid detection of an anomaly caused by external events of the particular location. In addition to the detection sites, LIGO also includes two primary university research centers; MIT and Caltech. The Initial LIGO project, which used the very first interferometers built for observations, was conducted from 2002 to 2010. During this period, not a single detection was made. Later on, an upgraded version of the same, The Advanced LIGO project was installed over the span of four years, between 2010 to 2014. Within days of initiating observations with newly installed equipment, LIGO made the first successful detection of gravitational waves on September 14, 2015. The gravitational waves detected, were known to have originated from the collision of two black holes in a binary system, which is 1.3 billion years away. Seeing it’s success over years, the co-founders were awarded a Nobel Prize in the year 2017.

Currently, LIGO employs around 40 people for each construction site. They are engineers, technicians and scientists who operate and overlook the functioning of the instruments and systems. On the other hand, LIGO engineers at Caltech and MIT work on improving LIGO’s stability and sensitivity, and the LIGO physicists and astrophysicists interpret the nature of the detected gravitational waves. LIGO’s mission is to open a window for a new area of scientific research on gravitational-wave astrophysics. It also conducts public outreach programs in order to provide opportunities for the scientific community to contribute to the enhancement of detectors, observation and data analysis. 

